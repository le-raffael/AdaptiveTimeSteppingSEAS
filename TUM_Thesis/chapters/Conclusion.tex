\chapter{Conclusion}
\label{chap:Conclusion}

In SEAS simulations, two tectonic plates are represented by elastostatic bodies that move in opposite directions. The fault is modeled as a 2D surface with a rate-and-state friction law acting as an algebraic constraint on the slip rate. The SEAS problem stated in \autoref{chap:BackgroundAndTheory} was identified as a semi-explicit differential algebraic equation (DAE) of index 1. In \autoref{chap:FirstExperiments}, a simple model on a single node where an analytic solution is available was set up to compare different timestep control strategies and to test the accuracy of the error estimates. The major part of the thesis considered a 2D domain with a 1D fault described in \autoref{chap:2DSEAS} where stresses and displacements are orthogonal to the surface. To solve the problem, four formulations of the time integration were developed and analyzed in \autoref{chap:44rmulations4SEAS}. All formulations were investigated for accuracy and scalability in \autoref{chap:Results} and it was concluded that two formulations perform better than the initial implementation of the SEAS problem. Finally, \autoref{chap:Extensions3D} showed that the formulations could be extended to a general 3D SEAS models with comparable performance results. \\

The first order ODE formulation integrates over the time derivatives of the slip $S$ and the state variable $\psi$. The slip rate $V$ is calculated iteratively at each evaluation of the right-hand side. It is compatible with explicit and implicit time integration schemes. In the extended DAE formulation, $V$ is added to the solution vector and the friction law is solved implicitly along with $S$ and $\psi$. In the compact DAE formulation, the slip rate is replaced by the first time derivative of the slip $\dot{S}$, which is obtained numerically. This allows us to remove the component $V$ from the solution vector and thus to consider a smaller problem. Both DAE formulations have to be solved with implicit methods. The SEAS problem can also be formulated as a 2nd order ODE by finding the time derivative of the slip rate, i.e. the slip acceleration. This last formulation has the advantage that the algebraic constraint disappears and does not need require to solve the elastic problem on the full domain at each evaluation of the right-hand side. It could be shown that the physical system behaves the same with all formulations. However, the compact DAE formulation is not suitable for the earthquake phase because the Newton iteration cannot reach a sufficiently accurate solution. \\

The timestep size is controlled with the elementary local error control. Explicit formulations are solved with the Dormand-Prince Runge-Kutta (RK-DP) scheme of 4th order with an embedded 5th order error estimate. Implicit formulations are solved with the backward differentiation formula (BDF) of order 1 to 6, where the order is adapted after each step to maximize the timestep size. To perform one timestep with implicit methods, a Newton iteration is required and at each Newton step, a linear system with the Jacobian matrix of the system has to be solved. For this task, the iterative GMRES method with a SOR preconditioner has proven to be well-suited as is much faster than using an expensive direct solver.  \\

Among first order formulations, it was observed that implicit methods keep similar timestep sizes when the number of fault elements increases, whereas explicit methods have to reduce them. This has a direct impact on the total number of timesteps to reach the end of the simulation and thus on the execution time. On the other hand, the costs of one implicit timestep increase with higher mesh resolutions as both the Newton and the GMRES iterations need on average more steps to reach the desired tolerance value. The costs of explicit methods are much better predictable as each timestep always requires six evaluations of the right-hand side in the RK-DP scheme and no linear system has to be solved. The implicit 1st order ODE and the DAE formulations produce almost identical timestep sizes but the GMRES iteration is more efficient with the DAE formulations. Therefore, for higher mesh resolutions, only the extended DAE formulation can compete with the explicit 1st order ODE formulation and is almost twice as fast for the highest considered mesh resolution. The compact DAE formulation might be even faster because it has fewer solution components, but it is only suited for the aseismic phase. \\

The explicit 2nd order ODE formulation has the fastest execution times. At first glance, it requires more timesteps than any other formulation because it solves a problem of higher order. However, this apparent deficit is largely compensated by the low execution costs of one timestep. Unlike first order formulations, it does not require solving the elastic problem at each right-hand side evaluation, which is usually the most expensive operation. For all mesh resolutions, it is about 10 times faster than the reference implementation. However, as an explicit formulation, it does not scale well with the problem size. The implicit 2nd order ODE formulation is not a suitable alternative because calculating and solving the Jacobian matrix is prohibitively expensive. Therefore, it might be that for finer meshes, the extended DAE formulation is more performant than the explicit 2nd order ODE. \\

The implemented program allows the user to specify separate solving parameters for the earthquake and aseismic phases. To reach a similar degree of accuracy, the tolerance for the time integration can be less strict in the earthquake phase. This is important because most of the simulation time is usually spent in the earthquake phase and tolerance values that are smaller than necessary increase the total number of timesteps without significant benefits. Furthermore, it is possible to switch the formulation and the time integration method between both phases. In most cases, there are no major advantages to do so, because all conclusions about time step sizes and scalability hold alike everywhere. A sensible combination would be to use the compact DAE in the aseismic phase and the extended DAE in the earthquake phase. Both formulations behave very similarly but the compact DAE is cheaper to apply. However, since it cannot be used in the earthquake phase, the extended DAE is needed there instead. \\

Overall, the explicit 2nd order ODE formulation is to be preferred, except for very high mesh resolutions, where an implicit DAE formulation has to be considered. Only for very short simulation lengths, when the total number of timesteps is of the order of the number of fault elements, the explicit 1st order ODE formulation is best-suited because it does not need an expensive precomputation of the partial derivatives of the friction law. A careful definition of the time integration tolerances for the aseismic and earthquake phases is essential for good performance. \\

Based on the results from the 2D scenario, only the implicit 1st order ODE, the extended DAE and the explicit 2nd order ODE formulations were added to the existing 3D code framework. \\

For further investigations, a new combined formulation based on the extended DAE and 2nd order ODE formulations was outlined in \autoref{sec:Results_towardsAnIdealModel}, which likely would have increased the execution speed by an important factor, but was not implemented nor tested. To achieve meaningful results on 3D domains with a sufficient fault resolution, an efficient parallelization of {\ttfamily tandem} is the next logical step. Further, the quasi-elastostatic assumption for the tectonic plates is only well-suited for the aseismic phase. During an earthquake abrupt movements at the fault induce elastic waves that may provoke major deformations at more distant points on the fault or inside the tectonic plates. To take care of these waves an elastodynamic model with a hyperbolic PDE is required. In the future, {\ttfamily tandem} could be coupled with existing software such as {\ttfamily SeisSol} in the earthquake phase to obtain a more accurate representation of the reality.
