\chapter{Introduction}
Earthquakes have always been a threat to human settlements and for centuries scientists dedicated their work to understand them. It is only after Alfred Wegener postulated the drift of tectonic plates in 1912 that a comprehensive model became available \cite{Wegener}. The earth's crust consists of several rigid plates driven by earth mantle convection and the friction between them is at the origin of earthquakes \cite{introductionGeophysics}. The scientific analysis of the period preceding the rupture process, also referred as nucleation, is still at its beginning. Fault monitoring to associate physical anomalies with imminent earthquakes \cite{AnomaliesNucleation,MagneticNucleation} or small-scale experimental setups for the material weakening and rock rupture preceding the nucleation \cite{ExperimentalNucleation} have been conducted, but a lot more research has to be done to fully understand the rupture. \\
Numerical simulations have been successful to reproduce singular earthquake events \cite{SeissolBigSimulation} and for the dynamic rupture itself \cite{DynamicRupture}. However, the scientific community is still lacking an accurate model and a performant simulation framework to understand how the fault creep in the aseismic phase and earthquakes are linked. In 2018, the Southern California Earthquake Center called to submit a community code for simulating sequences of earthquakes and aseismic slip (SEAS) on two benchmark problems \cite{BP1-Benchmark}. One research group to work on submission is the Truly Extended Earthquake Rupture (TEAR) project\footnote{\href{https://www.tear-erc.eu/}{https://www.tear-erc.eu/}}, hosted by the Institute for Geophysics at the Ludwig-Maximilians Universität in Munich, which aims to develop a new complete model of tectonic slip on faults, combining field observations, laboratory data and numerical simulations. \\
Within this framework, the software project {\ttfamily tandem}\footnote{\href{https://github.com/TEAR-ERC/tandem}{https://github.com/TEAR-ERC/tandem}} is currently under development to simulate SEAS models. At the time of writing this thesis, it only provided a sequential implementation. The two tectonic plates are represented by elastostatic bodies and interact with each other through a friction law, resulting in a differential-algebraic equation (DAE). A major challenge is to tackle the issue of different time scales. During the aseismic phase, stress slowly builds up at the interface for several decades until an earthquake occurs, allowing for timestep sizes above $10^6$s, or about ten days. On the other hand, the earthquake is characterized by rapid structural changes on the fault in under a minute, and timesteps of the order of milliseconds are needed for high accuracy. \\
In the scope of this thesis, four possible mathematical formulations of the DAE problem, that allow for explicit and implicit numerical schemes, are described and analyzed in depth. The overall performance, scalability and time-adaptivity of different time integration methods with these four formulations are then investigated. Three scenarios are considered: first, a one-dimensional model problem where an analytic solution is available is used to evaluate the overall accuracy and the error estimates of the different numerical schemes. Next, a two-dimensional, antiplane shear model is used to perform most numerical experiments. Finally, the findings from the simplified 2D model are generalized to a 3D model, which can represent arbitrary terrain profiles and is the closest to simulating a real-world tectonic fault. \\
The thesis is structured in three parts, which roughly follow the three scenarios. The fundamental theory of SEAS simulations and numerical solvers for DAE problems are stated in \autoref{chap:BackgroundAndTheory} and the one-dimensional problem is presented and analyzed in \autoref{chap:FirstExperiments}. The 2D antiplane shear problem is described in \autoref{chap:2DSEAS}, followed by the four formulations of the DAE problem and a discussion of their respective peculiarities in \autoref{chap:44rmulations4SEAS} and an accuracy and performance analysis in \autoref{chap:Results}. How the four formulations can be adapted for arbitrary three-dimensional domains and whether the conclusions from the 2D scenario still hold is treated in \autoref{chap:Extensions3D}. The thesis is concluded in \autoref{chap:Conclusion} with a summary of the findings and an outlook to possible future work.


