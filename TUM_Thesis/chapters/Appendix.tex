% !TEX root = ../main.tex
	
\chapter{Butcher tableaus of the used Runge-Kutta schemes}
\label{apx:ButcherTableaus}

\section{2nd order Bogacki-Shampine method}
It is a 2nd order scheme with a 3rd order embedded method for the error estimate developped in 1989 \cite{RK-BogackiShampine}. It is well-suited if a rough error estimation is sufficient. The Butcher tableau reads:
\begin{center}
	\begin{tabular}{c | c c c c}
		0 & & & & \\
		1/2 & 1/2 & & & \\
		3/4 & 0 & 3/4 & & \\
		1 & 2/9 & 1/3 & 4/9 & \\ \hline
		& 2/9 & 1/3 & 4/9 & 0 \\
		& 7/24 & 1/4 & 2/3 & 1/8
	\end{tabular}
\end{center}

\section{4th order Fehlberg method}
It is a 4th order scheme with a 5th order embedded method for the error estimate developped in 1969 \cite{RK-Fehlberg}. It gives a more accurate error estimate than the RK-BS scheme and converges with 4th order. The Butcher tableau reads:

\begin{center}
	\begin{tabular}{c | c c c c c c}
		0 & & & & & & \\
		1/4 & 1/4 & & & & & \\
		3/8 & 3/32 & 9/32 & & & &\\
		12/13 & 1932/2197 &	−7200/2197 & 7296/2197 & & & \\
		1 & 439/216 & −8 & 3680/513 & −845/4104 & & \\
		1/2 & −8/27 & 2 & −3544/2565 & 1859/4104 & −11/40 	& \\ \hline
		& 16/135 & 0 & 6656/12825 & 28561/56430 & −9/50 & 2/55 \\ 
		& 25/216 & 0 & 1408/2565 & 2197/4104 & −1/5 & 0 
	\end{tabular}
\end{center}


\section{5th order Dormand-Prince method}
It is a 5th order scheme with a 4th order embedded method for the error estimate developped in 1969 \cite{RK-DormandPrince}. Unlike the Fehlberg scheme, it minimizes the error of the 5th order solution. The Butcher tableau reads:

\begin{center}
	\begin{tabular}{c | c c c c c c c}
		0 & & & & & & & \\
		1/5 & 1/5 & & & & & & \\
		3/10 & 3/40 & 9/40 & & & & & \\
		4/5 & 44/45 & −56/15 & 32/9 & & & & \\
		8/9 & 19372/6561 & −25360/2187 & 64448/6561 & −212/729 & & & \\
		1 & 9017/3168 & −355/33 & 46732/5247 & 49/176 & −5103/18656 & & \\
		1 & 35/384 & 0 & 500/1113 & 125/192 & −2187/6784 & 11/84 & \\ \hline	
		& 35/384 & 0 & 500/1113 & 125/192 & −2187/6784 & 11/84 & 0 \\
		& 5179/57600 & 0 & 7571/16695 & 393/640 & −92097/339200 & 187/2100 & 1/40 
	\end{tabular}
\end{center}


\chapter{Alternative derivation of the BDF method}
\label{apx:BDF_derivation_Taylor}
The first order BDF-scheme corresponds to the backward Euler method in \autoref{eq:BDF_coeffs_1st_order}.
\begin{equation}
\label{eq:BDF_coeffs_1st_order}
\psi_{n+1} = \psi_n + h_{n}f(\psi_{n+1},V_{n+1})
\end{equation}
Next, we try to derive the second order BDF-scheme. Because of the adaptive time-stepping, the traditional coefficients of the BDF2 scheme cannot be used, but will be dependent of the current and previous timestep sizes $h_{n+1}$ and $h_n$. To find these coefficients, the Taylor polynomials of $\psi_n$ and $\psi_{n+1}$ are evaluated with respect to the unknown $\psi_{n+2}$. 
\begin{align}
\label{eq:taylor-polynomialBDF1(1)}
\psi_{n} &= \psi_{n+2} - (h_{n} + h_{n+1})\frac{d\psi_{n+2}}{dt} + \frac{(h_{n} + h_{n+1})^2}{2}\frac{d^2\psi_{n+2}}{dt^2} + \mathcal{O}\left((h_{n} + h_{n+1})^3\right) \\
\label{eq:taylor-polynomialBDF1(2)}
\psi_{n+1} &= \psi_{n+2} - h_{n+1}\frac{d\psi_{n+2}}{dt} + \frac{h_{n+1}^2}{2}\frac{d^2\psi_{n+2}}{dt^2} + \mathcal{O}\left(h_{n+1}^3\right)
\end{align}
The idea is to add equations (\ref{eq:taylor-polynomialBDF1(1)}) and (\ref{eq:taylor-polynomialBDF1(2)}), where the latter is multiplied by a factor $\alpha$ in a way that the second-derivative term drops. The addition of the two Taylor-expansions yields: 
\begin{equation}
\psi_{n} + \alpha \psi_{n+1} = (1+\alpha)\psi_{n+2} - \left(h_{n} + (1+\alpha)h_{n+1}\right)\frac{d\psi_{n+2}}{dt} + \frac{(h_{n} + h_{n+1})^2+\alpha h_{n+1}^2}{2}\frac{d^2\psi_{n+2}}{dt^2} + \mathcal{O}\left(h^3\right) \\
\end{equation}
By the choice of $\alpha$ in \autoref{eq:alpha}, the coefficient in front of the second time derivative of $\psi_{n+2}$ vanishes and the second order time adaptive BDF method is given by: \autoref{eq:BDF_coeffs_2nd_order}. 
\begin{align}
\label{eq:alpha}
\alpha &= -\left(\frac{h_n}{h_{n+1}}\right)^2 - 2\frac{h_n}{h_{n+1}} - 1 \\
\label{eq:BDF_coeffs_2nd_order}
\psi_n + \alpha \psi_{n+1} -(1+\alpha)\psi_{n+2} &= -\left(h_n + (1+\alpha)h_{n+1}\right)f(\psi_{n+2}, V_{n+2})
\end{align}
Analogously, the time-adapative third-order BDF3 scheme is given with the coefficients: 
\begin{align}
\alpha &= -\frac{\left(h_n+h_{n+1}\right)\left(h_n+h_{n+1}+h_{n+2}\right)^2}
{h_{n+1}\left(h_{n+1}+h_{n+2}\right)^2} \\
\beta &= \frac{h_n\left(h_n+h_{n+1}+h_{n+2}\right)^2}
{h_{n}^2h_{n+1}}
\end{align}
\begin{equation}
\label{eq:BDF_coeffs_3rd_order}
\psi_n + \alpha \psi_{n+1} + \beta \psi_{n+2} -(1+\alpha+\beta)\psi_{n+3} = \left(h_n + (1+\alpha)h_{n+1} + (1+\alpha+\beta)h_{n+2}\right)f(\psi_{n+3},V_{n+3})
\end{equation}

\chapter{Reduced Jacobian system for the extended DAE formulation}
\label{apx:ReducedJacobianExtendedDAE}
To efficiently apply iterative solvers for the extended DAE formulation, a blockwise Gaussian-elimination on the sparse components of the Jacobian matrix can be calculated such that the iterative solvers are only applied on a dense subsystem. For a general 2D problem, the Jacobian system takes the form: 
\begin{equation}
\begin{pmatrix}
\mathbf{J}_{11} & \mathbf{0}      & \mathbf{J}_{13} \\
\mathbf{0}        & \mathbf{J}_{22} & \mathbf{J}_{23}   \\
\mathbf{B}      & \mathbf{J}_{32} & \mathbf{J}_{33}   \\
\end{pmatrix}\begin{pmatrix}
x_1 \\ x_2 \\ x_3 
\end{pmatrix} = \begin{pmatrix}
b_1 \\ b_2 \\ b_3 
\end{pmatrix}
\end{equation}
The sparse components are in the submatrices $\mathbf{J}_{ij}$ and the dense parts are in $\mathbf{B}$. After the Gaussian elimination, we get:
\begin{align}
\begin{pmatrix}
\mathbf{B} - \mathbf{J}_{33}\mathbf{J}_{13}^{-1}\mathbf{J}_{11} + \mathbf{J}_{32}\mathbf{J}_{22}^{-1}\mathbf{J}_{23}\mathbf{J}_{13}^{-1}\mathbf{J}_{11} & \mathbf{0} & \mathbf{0} \\
-\mathbf{J}_{23}\mathbf{J}_{13}^{-1}\mathbf{J}_{11} & \mathbf{J}_{22} & \mathbf{0}   \\
\mathbf{J}_{11} & \mathbf{0} & \mathbf{J}_{13} \\
\end{pmatrix}\begin{pmatrix}
x_1 \\ x_2 \\ x_3 
\end{pmatrix} =  \qquad\qquad\qquad\qquad\nonumber \\ \qquad\qquad\qquad\qquad\qquad\qquad\qquad
\begin{pmatrix}
b_3 - \mathbf{J}_{33}\mathbf{J}_{13}^{-1}b_1 - \mathbf{J}_{32}\mathbf{J}_{22}^{-1}\left(b_2 - \mathbf{J}_{23}\mathbf{J}_{13}^{-1}b_1\right) 
\\ b_2 - \mathbf{J}_{23}\mathbf{J}_{13}^{-1}b_1 
\\ b_1 
\end{pmatrix}
\end{align}
The upper-left block form the reduced system which is solved iteratively. The other solution components are then calculated by an easy backward substitution.
