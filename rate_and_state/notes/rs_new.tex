\documentclass[paper=a4]{scrartcl}
\usepackage{amsmath}
\usepackage{amssymb}
\usepackage{tensor}
\usepackage{stmaryrd}
\usepackage[utf8]{inputenc}
\usepackage[T1]{fontenc}
\usepackage{csquotes}
\usepackage{booktabs}
\usepackage{tikz}
\usepackage{cleveref}

\renewcommand{\vec}{\boldsymbol}
\newcommand{\avg}[1]{\{\!\{#1\}\!\}}
\newcommand{\jump}[1]{\llbracket#1\rrbracket}

% nice differentials
% http://www.tug.org/TUGboat/Articles/tb18-1/tb54becc.pdf
\makeatletter
\providecommand*{\dd}%
  {\@ifnextchar^{\DIfF}{\DIfF^{}}}
\def\DIfF^#1{%
  \mathop{\mathrm{\mathstrut d}}%
    \nolimits^{#1}\gobblespace}
\def\gobblespace{%
  \futurelet\diffarg\opspace}
\def\opspace{%
  \let\DiffSpace\!%
  \ifx\diffarg(%
    \let\DiffSpace\relax
  \else
    \ifx\diffarg[%
      \let\DiffSpace\relax
    \else
      \ifx\diffarg\{%
        \let\DiffSpace\relax
      \fi\fi\fi\DiffSpace}
\providecommand*{\deriv}[3][]{%
  \frac{\dd^{#1}#2}{\dd#3^{#1}}}
\providecommand*{\pderiv}[3][]{%
  \frac{\partial^{#1}#2}%
    {\partial#3^{#1}}}
\makeatother

\DeclareMathOperator{\arcsinh}{arcsinh}

\title{Rate and state}
\author{Carsten Uphoff}

\begin{document}

\maketitle

\section{Test problem}
\begin{align}
    0 &= \tau - \sigma_n a \arcsinh\left(
        \dfrac{V}{2V_0}\exp\left(\dfrac{\psi}{a}\right)
    \right) - \eta V\\
    \deriv{\psi}{t} &= \dfrac{bV_0}{L}\left(\exp\left(\dfrac{f_0-\psi}{b}\right) - \dfrac{V}{V_0}\right)
\end{align}

\section{Manufactured solution}
Let
\begin{equation}
    V^*(t) = \frac{1}{\pi}\left(\arctan\left(\frac{t-t_e}{t_w}\right) + \frac{\pi}{2}\right)
\end{equation}

\begin{equation}
    \deriv{V^*}{t} = \frac{1}{\pi t_w}\frac{1}{\left(\frac{t-t_e}{t_w}\right)^2 + 1}
\end{equation}

Solve friction law for $\psi$:
\begin{equation}
    \psi^*(t) =
        a \log\left(
            \frac{2V_0}{V^*}\sinh\left(
                \frac{\tau^* - \eta V^*}{a s_n}
            \right)
        \right)
\end{equation}

\begin{equation}
    \deriv{\psi^*}{t} =
    \frac{a\left(
        \frac{2V_0}{Vas_n}\cosh\left(
                \frac{\tau^* - \eta V^*}{a s_n}
            \right)\left(\deriv{\tau^*}{t} - \eta\deriv{V^*}{t}\right)
        -\frac{2V_0\deriv{V^*}{t}}{(V^*)^2}\sinh\left(
                \frac{\tau^* - \eta V^*}{a s_n}
            \right)
    \right)}{\frac{2V_0}{V^*}\sinh\left(
                \frac{\tau^* - \eta V^*}{a s_n}
            \right)}
\end{equation}

State ODE:
\begin{equation}
    \deriv{\psi}{t} = \dfrac{bV_0}{L}\left(\exp\left(\dfrac{f_0-\psi}{b}\right) - \dfrac{V}{V_0}\right)
    - \dfrac{bV_0}{L}\left(\exp\left(\dfrac{f_0-\psi^*}{b}\right) - \dfrac{V^*}{V_0}\right) + \deriv{\psi*}{t}
\end{equation}

\section{Bounds}

Assume $\sigma_n, \tau > 0$. Let $V=\tau/\eta$. Then
\begin{equation}
    f(\tau/\eta) = - \sigma_n a \arcsinh\left(
        \dfrac{\tau/\eta}{2V_0}\exp\left(\dfrac{\psi}{a}\right)
    \right) < 0
\end{equation}
Let $V=-\tau/\eta$.
\begin{equation}
    f(-\tau/\eta) = 2\tau - \sigma_n a \arcsinh\left(
        \dfrac{-\tau/\eta}{2V_0}\exp\left(\dfrac{\psi}{a}\right)
    \right) > 0
\end{equation}
(If $\tau < 0$ the situation reverses, if $\tau = 0$ then $V=0$ is a zero of $f$.)

Therefore we may use $[-\tau/\eta, \tau/\eta]$ as bisection interval.

\end{document}
