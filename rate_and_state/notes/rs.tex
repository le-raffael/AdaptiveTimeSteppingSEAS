\documentclass[paper=a4]{scrartcl}
\usepackage{amsmath}
\usepackage{amssymb}
\usepackage{tensor}
\usepackage{stmaryrd}
\usepackage[utf8]{inputenc}
\usepackage[T1]{fontenc}
\usepackage{csquotes}
\usepackage{booktabs}
\usepackage{tikz}
\usepackage{cleveref}

\renewcommand{\vec}{\boldsymbol}
\newcommand{\avg}[1]{\{\!\{#1\}\!\}}
\newcommand{\jump}[1]{\llbracket#1\rrbracket}

% nice differentials
% http://www.tug.org/TUGboat/Articles/tb18-1/tb54becc.pdf
\makeatletter
\providecommand*{\dd}%
  {\@ifnextchar^{\DIfF}{\DIfF^{}}}
\def\DIfF^#1{%
  \mathop{\mathrm{\mathstrut d}}%
    \nolimits^{#1}\gobblespace}
\def\gobblespace{%
  \futurelet\diffarg\opspace}
\def\opspace{%
  \let\DiffSpace\!%
  \ifx\diffarg(%
    \let\DiffSpace\relax
  \else
    \ifx\diffarg[%
      \let\DiffSpace\relax
    \else
      \ifx\diffarg\{%
        \let\DiffSpace\relax
      \fi\fi\fi\DiffSpace}
\providecommand*{\deriv}[3][]{%
  \frac{\dd^{#1}#2}{\dd#3^{#1}}}
\providecommand*{\pderiv}[3][]{%
  \frac{\partial^{#1}#2}%
    {\partial#3^{#1}}}
\makeatother

\DeclareMathOperator{\arcsinh}{arcsinh}

\title{Rate and state}
\author{Carsten Uphoff}

\begin{document}

\maketitle

\section{Test problem}
\begin{align}
    0 &= \tau - \sigma_n a \arcsinh\left(
        \dfrac{V}{2V_0}\exp\left(\dfrac{f_0 + b\log(V_0\psi/L)}{a}\right)
    \right) - \eta V\\
    \deriv{\psi}{t} &= 1 - \dfrac{V\psi}{L}
\end{align}

\section{Manufactured solution}
Let
\begin{equation}
    V^*(t) = \frac{1}{\pi}\left(\arctan\left(\frac{t-t_e}{t_w}\right) + \frac{\pi}{2}\right)
\end{equation}

\begin{equation}
    \deriv{V^*}{t} = \frac{1}{\pi t_w}\frac{1}{\left(\frac{t-t_e}{t_w}\right)^2 + 1}
\end{equation}

Solve friction law for $\psi$:
\begin{equation}
    \psi^*(t) = \frac{L}{V_0} \exp\left(
        \frac{a}{b} \log\left(
            \frac{2V_0}{V}\sinh\left(
                \frac{\tau - \eta V}{a s_n}
            \right)
        \right)
        - \frac{f_0}{b}
    \right)
\end{equation}

\begin{multline}
    \deriv{\psi^*}{t} = \frac{aV\frac{L}{V_0} \exp\left(
        \frac{a}{b} \log\left(
            \frac{2V_0}{V}\sinh\left(
                \frac{\tau - \eta V}{a s_n}
            \right)
        \right)
        - \frac{f_0}{b}
    \right)}{
        2bV_0\sinh\left(
            \frac{\tau - \eta V}{a s_n}
        \right)
    }\\ \times
    \left(
        \frac{2V_0}{Vas_n}\cosh\left(
            \frac{\tau - \eta V}{a s_n}
        \right)\left(\deriv{\tau}{t} - \eta \deriv{V}{t}\right)
        -\frac{2V_0 \deriv{V}{t}}{V^2}\sinh\left(
            \frac{\tau - \eta V}{a s_n}
        \right)
    \right)
\end{multline}

State ODE:
\begin{equation}
    \deriv{\psi}{t} = 1 - \frac{V\psi}{L} - \left(1 - \frac{V^*\psi^*}{L}\right) + \deriv{\psi*}{t}
\end{equation}

\section{Bounds}

Assume $\sigma_n, \tau > 0$. Let $V=\tau/\eta$. Then
\begin{equation}
    f(\tau/\eta) = - \sigma_n a \arcsinh\left(
        \dfrac{\tau/\eta}{2V_0}\exp\left(\dfrac{f_0 + b\log(V_0\psi/L}{a}\right)
    \right) < 0
\end{equation}
Let $V=-\tau/\eta$.
\begin{equation}
    f(-\tau/\eta) = 2\tau - \sigma_n a \arcsinh\left(
        \dfrac{-\tau/\eta}{2V_0}\exp\left(\dfrac{f_0 + b\log(V_0\psi/L}{a}\right)
    \right) > 0
\end{equation}
(If $\tau < 0$ the situation reverses, if $\tau = 0$ then $V=0$ is a zero of $f$.)

Therefore we may use $[-\tau/\eta, \tau/\eta]$ as bisection interval.

\section{Steady state}
Let 
\begin{align}
    0 &= \tau - \sigma_n a \arcsinh\left(
        \dfrac{V}{2V_0}\exp\left(\dfrac{f_0 + b\log(V_0\psi/L)}{a}\right)
    \right) \\
    0 &= 1 - \dfrac{V\psi}{L}
\end{align}
Steady state is
\begin{equation}
    \psi_{ss}(V, \tau) = \frac{L}{V}
\end{equation}
\begin{equation}
    \tau_{ss}(V) = \sigma_n a\arcsinh\left(
        \frac{V}{2V_0}\exp\left(\frac{b\log(V_0/V) + f_0}{a}\right)
    \right)
\end{equation}
\begin{equation}
    \deriv{\tau_{ss}}{V} = \sigma_n
    \dfrac{\exp\left(\frac{b\log(V_0/V) + f_0}{a}\right)
           (a - b)}
          {2V_0\sqrt{\exp\left(\frac{2b\log(V_0/V) + 2f_0}{a}\right)V^2/(4V_0^2) + 1}}
\end{equation}

\section{Critical sizes (Lapusta)}
Take
\begin{align}
    0 &= \tau - \sigma_n \left(f_0 + a\log\dfrac{V}{V_0} + b\log\dfrac{V_0\psi}{L}\right) \\
    0 &= 1 - \dfrac{V\psi}{L}
\end{align}
Steady state is
\begin{align}
    \psi_{ss}(V) &= \frac{L}{V} \\
    \tau_{ss}(V) &= \sigma_n \left(f_0 + a\log\dfrac{V}{V_0} + b\log\dfrac{V_0}{V}\right) \\
    \deriv{\tau{ss}}{V} &= \sigma_n \left(a\dfrac{1}{V} - b\dfrac{1}{V}\right)
\end{align}
Critical size is
\begin{equation}
    h^* = \dfrac{\gamma\mu L}{\sigma_n(b-a)}
\end{equation}
For the time-step we need
\begin{align}
    A^* &= \sigma_n a \\
    B^* - A^* &= \sigma_n(a-b) \\
    \chi &= \dfrac{1}{4}\dfrac{B^*-A^*}{A^*}\left(\dfrac{h^*}{h} - 1\right)^2 - \dfrac{h^*}{h} \\
    \Delta t &< \left\{\begin{array}{rcl}
        \dfrac{L}{V}\dfrac{A^*}{(B^*-A^*)(h^*/h-1)} & \text{ if } & \chi > 0 \\
        \dfrac{L}{V}(1-h/h^*) & \text{ if } & \chi < 0
    \end{array}\right.
\end{align}

\section{Time-stepping elastodynamics}
Let the semi-discrete form be given by
\begin{equation}
    \ddot{u}_p = b_p(\vec{S}(t)) - \hat{K}_{pq}u_q
\end{equation}
First order form is
\begin{align*}
    \dot{u}_p &= v_p \\
    \dot{v}_p &= b_p(\vec{S}(t)) - \hat{K}_{pq}u_q
\end{align*}
Lets compactly write
\begin{align*}
    \dot{y}_p &= c_p(\vec{S}(t)) - G_{pq}y_p,
\end{align*}
where
\begin{equation*}
    \vec{y} = \begin{pmatrix}\vec{u} \\ \vec{v}\end{pmatrix}, \quad
    \vec{c} = \begin{pmatrix}0 \\ \vec{b}\end{pmatrix}, \quad
    G = \begin{pmatrix}
        0 & -I \\
        \hat{K} & 0
    \end{pmatrix}
\end{equation*}

The Runge-Kutta method is
\begin{align}
    y_p^{n+1} &= y_p^n + h b_j k_{pj} \\
    k_{pi} &= c_p(\vec{S}(t_n + hc_i)) - G_{pq} (1_iy_p^n + h a_{ij} k_{pj})
\end{align}
Collecting $k$ on the LHS gives
\begin{equation*}
    k_{pi} + h a_{ij} G_{pq} k_{pj} = c_p(\vec{S}(t_n + hc_i)) - 1_iG_{pq} y_p^n
\end{equation*}
Obviously,
\begin{equation*}
    \underbrace{(\delta_{ij} + h a_{ij})}_{=:\alpha_{ij}} G_{pq} k_{pj} = c_p(\vec{S}(t_n + hc_i)) - 1_iG_{pq} y_p^n
\end{equation*}
Multiply with $\alpha^{-1}_{si}$, which should exist (?), to obtain
\begin{equation*}
    G_{pq} k_{ps} = \alpha^{-1}_{si} c_p(\vec{S}(t_n + hc_i)) - (\alpha^{-1}_{si}1_i)G_{pq} y_p^n
\end{equation*}


\end{document}

